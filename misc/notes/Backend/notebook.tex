
% Default to the notebook output style

    


% Inherit from the specified cell style.




    
\documentclass[11pt]{article}

    
    
    \usepackage[T1]{fontenc}
    % Nicer default font (+ math font) than Computer Modern for most use cases
    \usepackage{mathpazo}

    % Basic figure setup, for now with no caption control since it's done
    % automatically by Pandoc (which extracts ![](path) syntax from Markdown).
    \usepackage{graphicx}
    % We will generate all images so they have a width \maxwidth. This means
    % that they will get their normal width if they fit onto the page, but
    % are scaled down if they would overflow the margins.
    \makeatletter
    \def\maxwidth{\ifdim\Gin@nat@width>\linewidth\linewidth
    \else\Gin@nat@width\fi}
    \makeatother
    \let\Oldincludegraphics\includegraphics
    % Set max figure width to be 80% of text width, for now hardcoded.
    \renewcommand{\includegraphics}[1]{\Oldincludegraphics[width=.8\maxwidth]{#1}}
    % Ensure that by default, figures have no caption (until we provide a
    % proper Figure object with a Caption API and a way to capture that
    % in the conversion process - todo).
    \usepackage{caption}
    \DeclareCaptionLabelFormat{nolabel}{}
    \captionsetup{labelformat=nolabel}

    \usepackage{adjustbox} % Used to constrain images to a maximum size 
    \usepackage{xcolor} % Allow colors to be defined
    \usepackage{enumerate} % Needed for markdown enumerations to work
    \usepackage{geometry} % Used to adjust the document margins
    \usepackage{amsmath} % Equations
    \usepackage{amssymb} % Equations
    \usepackage{textcomp} % defines textquotesingle
    % Hack from http://tex.stackexchange.com/a/47451/13684:
    \AtBeginDocument{%
        \def\PYZsq{\textquotesingle}% Upright quotes in Pygmentized code
    }
    \usepackage{upquote} % Upright quotes for verbatim code
    \usepackage{eurosym} % defines \euro
    \usepackage[mathletters]{ucs} % Extended unicode (utf-8) support
    \usepackage[utf8x]{inputenc} % Allow utf-8 characters in the tex document
    \usepackage{fancyvrb} % verbatim replacement that allows latex
    \usepackage{grffile} % extends the file name processing of package graphics 
                         % to support a larger range 
    % The hyperref package gives us a pdf with properly built
    % internal navigation ('pdf bookmarks' for the table of contents,
    % internal cross-reference links, web links for URLs, etc.)
    \usepackage{hyperref}
    \usepackage{longtable} % longtable support required by pandoc >1.10
    \usepackage{booktabs}  % table support for pandoc > 1.12.2
    \usepackage[inline]{enumitem} % IRkernel/repr support (it uses the enumerate* environment)
    \usepackage[normalem]{ulem} % ulem is needed to support strikethroughs (\sout)
                                % normalem makes italics be italics, not underlines
    

    
    
    % Colors for the hyperref package
    \definecolor{urlcolor}{rgb}{0,.145,.698}
    \definecolor{linkcolor}{rgb}{.71,0.21,0.01}
    \definecolor{citecolor}{rgb}{.12,.54,.11}

    % ANSI colors
    \definecolor{ansi-black}{HTML}{3E424D}
    \definecolor{ansi-black-intense}{HTML}{282C36}
    \definecolor{ansi-red}{HTML}{E75C58}
    \definecolor{ansi-red-intense}{HTML}{B22B31}
    \definecolor{ansi-green}{HTML}{00A250}
    \definecolor{ansi-green-intense}{HTML}{007427}
    \definecolor{ansi-yellow}{HTML}{DDB62B}
    \definecolor{ansi-yellow-intense}{HTML}{B27D12}
    \definecolor{ansi-blue}{HTML}{208FFB}
    \definecolor{ansi-blue-intense}{HTML}{0065CA}
    \definecolor{ansi-magenta}{HTML}{D160C4}
    \definecolor{ansi-magenta-intense}{HTML}{A03196}
    \definecolor{ansi-cyan}{HTML}{60C6C8}
    \definecolor{ansi-cyan-intense}{HTML}{258F8F}
    \definecolor{ansi-white}{HTML}{C5C1B4}
    \definecolor{ansi-white-intense}{HTML}{A1A6B2}

    % commands and environments needed by pandoc snippets
    % extracted from the output of `pandoc -s`
    \providecommand{\tightlist}{%
      \setlength{\itemsep}{0pt}\setlength{\parskip}{0pt}}
    \DefineVerbatimEnvironment{Highlighting}{Verbatim}{commandchars=\\\{\}}
    % Add ',fontsize=\small' for more characters per line
    \newenvironment{Shaded}{}{}
    \newcommand{\KeywordTok}[1]{\textcolor[rgb]{0.00,0.44,0.13}{\textbf{{#1}}}}
    \newcommand{\DataTypeTok}[1]{\textcolor[rgb]{0.56,0.13,0.00}{{#1}}}
    \newcommand{\DecValTok}[1]{\textcolor[rgb]{0.25,0.63,0.44}{{#1}}}
    \newcommand{\BaseNTok}[1]{\textcolor[rgb]{0.25,0.63,0.44}{{#1}}}
    \newcommand{\FloatTok}[1]{\textcolor[rgb]{0.25,0.63,0.44}{{#1}}}
    \newcommand{\CharTok}[1]{\textcolor[rgb]{0.25,0.44,0.63}{{#1}}}
    \newcommand{\StringTok}[1]{\textcolor[rgb]{0.25,0.44,0.63}{{#1}}}
    \newcommand{\CommentTok}[1]{\textcolor[rgb]{0.38,0.63,0.69}{\textit{{#1}}}}
    \newcommand{\OtherTok}[1]{\textcolor[rgb]{0.00,0.44,0.13}{{#1}}}
    \newcommand{\AlertTok}[1]{\textcolor[rgb]{1.00,0.00,0.00}{\textbf{{#1}}}}
    \newcommand{\FunctionTok}[1]{\textcolor[rgb]{0.02,0.16,0.49}{{#1}}}
    \newcommand{\RegionMarkerTok}[1]{{#1}}
    \newcommand{\ErrorTok}[1]{\textcolor[rgb]{1.00,0.00,0.00}{\textbf{{#1}}}}
    \newcommand{\NormalTok}[1]{{#1}}
    
    % Additional commands for more recent versions of Pandoc
    \newcommand{\ConstantTok}[1]{\textcolor[rgb]{0.53,0.00,0.00}{{#1}}}
    \newcommand{\SpecialCharTok}[1]{\textcolor[rgb]{0.25,0.44,0.63}{{#1}}}
    \newcommand{\VerbatimStringTok}[1]{\textcolor[rgb]{0.25,0.44,0.63}{{#1}}}
    \newcommand{\SpecialStringTok}[1]{\textcolor[rgb]{0.73,0.40,0.53}{{#1}}}
    \newcommand{\ImportTok}[1]{{#1}}
    \newcommand{\DocumentationTok}[1]{\textcolor[rgb]{0.73,0.13,0.13}{\textit{{#1}}}}
    \newcommand{\AnnotationTok}[1]{\textcolor[rgb]{0.38,0.63,0.69}{\textbf{\textit{{#1}}}}}
    \newcommand{\CommentVarTok}[1]{\textcolor[rgb]{0.38,0.63,0.69}{\textbf{\textit{{#1}}}}}
    \newcommand{\VariableTok}[1]{\textcolor[rgb]{0.10,0.09,0.49}{{#1}}}
    \newcommand{\ControlFlowTok}[1]{\textcolor[rgb]{0.00,0.44,0.13}{\textbf{{#1}}}}
    \newcommand{\OperatorTok}[1]{\textcolor[rgb]{0.40,0.40,0.40}{{#1}}}
    \newcommand{\BuiltInTok}[1]{{#1}}
    \newcommand{\ExtensionTok}[1]{{#1}}
    \newcommand{\PreprocessorTok}[1]{\textcolor[rgb]{0.74,0.48,0.00}{{#1}}}
    \newcommand{\AttributeTok}[1]{\textcolor[rgb]{0.49,0.56,0.16}{{#1}}}
    \newcommand{\InformationTok}[1]{\textcolor[rgb]{0.38,0.63,0.69}{\textbf{\textit{{#1}}}}}
    \newcommand{\WarningTok}[1]{\textcolor[rgb]{0.38,0.63,0.69}{\textbf{\textit{{#1}}}}}
    
    
    % Define a nice break command that doesn't care if a line doesn't already
    % exist.
    \def\br{\hspace*{\fill} \\* }
    % Math Jax compatability definitions
    \def\gt{>}
    \def\lt{<}
    % Document parameters
    \title{Post Requests}
    
    
    

    % Pygments definitions
    
\makeatletter
\def\PY@reset{\let\PY@it=\relax \let\PY@bf=\relax%
    \let\PY@ul=\relax \let\PY@tc=\relax%
    \let\PY@bc=\relax \let\PY@ff=\relax}
\def\PY@tok#1{\csname PY@tok@#1\endcsname}
\def\PY@toks#1+{\ifx\relax#1\empty\else%
    \PY@tok{#1}\expandafter\PY@toks\fi}
\def\PY@do#1{\PY@bc{\PY@tc{\PY@ul{%
    \PY@it{\PY@bf{\PY@ff{#1}}}}}}}
\def\PY#1#2{\PY@reset\PY@toks#1+\relax+\PY@do{#2}}

\expandafter\def\csname PY@tok@w\endcsname{\def\PY@tc##1{\textcolor[rgb]{0.73,0.73,0.73}{##1}}}
\expandafter\def\csname PY@tok@c\endcsname{\let\PY@it=\textit\def\PY@tc##1{\textcolor[rgb]{0.25,0.50,0.50}{##1}}}
\expandafter\def\csname PY@tok@cp\endcsname{\def\PY@tc##1{\textcolor[rgb]{0.74,0.48,0.00}{##1}}}
\expandafter\def\csname PY@tok@k\endcsname{\let\PY@bf=\textbf\def\PY@tc##1{\textcolor[rgb]{0.00,0.50,0.00}{##1}}}
\expandafter\def\csname PY@tok@kp\endcsname{\def\PY@tc##1{\textcolor[rgb]{0.00,0.50,0.00}{##1}}}
\expandafter\def\csname PY@tok@kt\endcsname{\def\PY@tc##1{\textcolor[rgb]{0.69,0.00,0.25}{##1}}}
\expandafter\def\csname PY@tok@o\endcsname{\def\PY@tc##1{\textcolor[rgb]{0.40,0.40,0.40}{##1}}}
\expandafter\def\csname PY@tok@ow\endcsname{\let\PY@bf=\textbf\def\PY@tc##1{\textcolor[rgb]{0.67,0.13,1.00}{##1}}}
\expandafter\def\csname PY@tok@nb\endcsname{\def\PY@tc##1{\textcolor[rgb]{0.00,0.50,0.00}{##1}}}
\expandafter\def\csname PY@tok@nf\endcsname{\def\PY@tc##1{\textcolor[rgb]{0.00,0.00,1.00}{##1}}}
\expandafter\def\csname PY@tok@nc\endcsname{\let\PY@bf=\textbf\def\PY@tc##1{\textcolor[rgb]{0.00,0.00,1.00}{##1}}}
\expandafter\def\csname PY@tok@nn\endcsname{\let\PY@bf=\textbf\def\PY@tc##1{\textcolor[rgb]{0.00,0.00,1.00}{##1}}}
\expandafter\def\csname PY@tok@ne\endcsname{\let\PY@bf=\textbf\def\PY@tc##1{\textcolor[rgb]{0.82,0.25,0.23}{##1}}}
\expandafter\def\csname PY@tok@nv\endcsname{\def\PY@tc##1{\textcolor[rgb]{0.10,0.09,0.49}{##1}}}
\expandafter\def\csname PY@tok@no\endcsname{\def\PY@tc##1{\textcolor[rgb]{0.53,0.00,0.00}{##1}}}
\expandafter\def\csname PY@tok@nl\endcsname{\def\PY@tc##1{\textcolor[rgb]{0.63,0.63,0.00}{##1}}}
\expandafter\def\csname PY@tok@ni\endcsname{\let\PY@bf=\textbf\def\PY@tc##1{\textcolor[rgb]{0.60,0.60,0.60}{##1}}}
\expandafter\def\csname PY@tok@na\endcsname{\def\PY@tc##1{\textcolor[rgb]{0.49,0.56,0.16}{##1}}}
\expandafter\def\csname PY@tok@nt\endcsname{\let\PY@bf=\textbf\def\PY@tc##1{\textcolor[rgb]{0.00,0.50,0.00}{##1}}}
\expandafter\def\csname PY@tok@nd\endcsname{\def\PY@tc##1{\textcolor[rgb]{0.67,0.13,1.00}{##1}}}
\expandafter\def\csname PY@tok@s\endcsname{\def\PY@tc##1{\textcolor[rgb]{0.73,0.13,0.13}{##1}}}
\expandafter\def\csname PY@tok@sd\endcsname{\let\PY@it=\textit\def\PY@tc##1{\textcolor[rgb]{0.73,0.13,0.13}{##1}}}
\expandafter\def\csname PY@tok@si\endcsname{\let\PY@bf=\textbf\def\PY@tc##1{\textcolor[rgb]{0.73,0.40,0.53}{##1}}}
\expandafter\def\csname PY@tok@se\endcsname{\let\PY@bf=\textbf\def\PY@tc##1{\textcolor[rgb]{0.73,0.40,0.13}{##1}}}
\expandafter\def\csname PY@tok@sr\endcsname{\def\PY@tc##1{\textcolor[rgb]{0.73,0.40,0.53}{##1}}}
\expandafter\def\csname PY@tok@ss\endcsname{\def\PY@tc##1{\textcolor[rgb]{0.10,0.09,0.49}{##1}}}
\expandafter\def\csname PY@tok@sx\endcsname{\def\PY@tc##1{\textcolor[rgb]{0.00,0.50,0.00}{##1}}}
\expandafter\def\csname PY@tok@m\endcsname{\def\PY@tc##1{\textcolor[rgb]{0.40,0.40,0.40}{##1}}}
\expandafter\def\csname PY@tok@gh\endcsname{\let\PY@bf=\textbf\def\PY@tc##1{\textcolor[rgb]{0.00,0.00,0.50}{##1}}}
\expandafter\def\csname PY@tok@gu\endcsname{\let\PY@bf=\textbf\def\PY@tc##1{\textcolor[rgb]{0.50,0.00,0.50}{##1}}}
\expandafter\def\csname PY@tok@gd\endcsname{\def\PY@tc##1{\textcolor[rgb]{0.63,0.00,0.00}{##1}}}
\expandafter\def\csname PY@tok@gi\endcsname{\def\PY@tc##1{\textcolor[rgb]{0.00,0.63,0.00}{##1}}}
\expandafter\def\csname PY@tok@gr\endcsname{\def\PY@tc##1{\textcolor[rgb]{1.00,0.00,0.00}{##1}}}
\expandafter\def\csname PY@tok@ge\endcsname{\let\PY@it=\textit}
\expandafter\def\csname PY@tok@gs\endcsname{\let\PY@bf=\textbf}
\expandafter\def\csname PY@tok@gp\endcsname{\let\PY@bf=\textbf\def\PY@tc##1{\textcolor[rgb]{0.00,0.00,0.50}{##1}}}
\expandafter\def\csname PY@tok@go\endcsname{\def\PY@tc##1{\textcolor[rgb]{0.53,0.53,0.53}{##1}}}
\expandafter\def\csname PY@tok@gt\endcsname{\def\PY@tc##1{\textcolor[rgb]{0.00,0.27,0.87}{##1}}}
\expandafter\def\csname PY@tok@err\endcsname{\def\PY@bc##1{\setlength{\fboxsep}{0pt}\fcolorbox[rgb]{1.00,0.00,0.00}{1,1,1}{\strut ##1}}}
\expandafter\def\csname PY@tok@kc\endcsname{\let\PY@bf=\textbf\def\PY@tc##1{\textcolor[rgb]{0.00,0.50,0.00}{##1}}}
\expandafter\def\csname PY@tok@kd\endcsname{\let\PY@bf=\textbf\def\PY@tc##1{\textcolor[rgb]{0.00,0.50,0.00}{##1}}}
\expandafter\def\csname PY@tok@kn\endcsname{\let\PY@bf=\textbf\def\PY@tc##1{\textcolor[rgb]{0.00,0.50,0.00}{##1}}}
\expandafter\def\csname PY@tok@kr\endcsname{\let\PY@bf=\textbf\def\PY@tc##1{\textcolor[rgb]{0.00,0.50,0.00}{##1}}}
\expandafter\def\csname PY@tok@bp\endcsname{\def\PY@tc##1{\textcolor[rgb]{0.00,0.50,0.00}{##1}}}
\expandafter\def\csname PY@tok@fm\endcsname{\def\PY@tc##1{\textcolor[rgb]{0.00,0.00,1.00}{##1}}}
\expandafter\def\csname PY@tok@vc\endcsname{\def\PY@tc##1{\textcolor[rgb]{0.10,0.09,0.49}{##1}}}
\expandafter\def\csname PY@tok@vg\endcsname{\def\PY@tc##1{\textcolor[rgb]{0.10,0.09,0.49}{##1}}}
\expandafter\def\csname PY@tok@vi\endcsname{\def\PY@tc##1{\textcolor[rgb]{0.10,0.09,0.49}{##1}}}
\expandafter\def\csname PY@tok@vm\endcsname{\def\PY@tc##1{\textcolor[rgb]{0.10,0.09,0.49}{##1}}}
\expandafter\def\csname PY@tok@sa\endcsname{\def\PY@tc##1{\textcolor[rgb]{0.73,0.13,0.13}{##1}}}
\expandafter\def\csname PY@tok@sb\endcsname{\def\PY@tc##1{\textcolor[rgb]{0.73,0.13,0.13}{##1}}}
\expandafter\def\csname PY@tok@sc\endcsname{\def\PY@tc##1{\textcolor[rgb]{0.73,0.13,0.13}{##1}}}
\expandafter\def\csname PY@tok@dl\endcsname{\def\PY@tc##1{\textcolor[rgb]{0.73,0.13,0.13}{##1}}}
\expandafter\def\csname PY@tok@s2\endcsname{\def\PY@tc##1{\textcolor[rgb]{0.73,0.13,0.13}{##1}}}
\expandafter\def\csname PY@tok@sh\endcsname{\def\PY@tc##1{\textcolor[rgb]{0.73,0.13,0.13}{##1}}}
\expandafter\def\csname PY@tok@s1\endcsname{\def\PY@tc##1{\textcolor[rgb]{0.73,0.13,0.13}{##1}}}
\expandafter\def\csname PY@tok@mb\endcsname{\def\PY@tc##1{\textcolor[rgb]{0.40,0.40,0.40}{##1}}}
\expandafter\def\csname PY@tok@mf\endcsname{\def\PY@tc##1{\textcolor[rgb]{0.40,0.40,0.40}{##1}}}
\expandafter\def\csname PY@tok@mh\endcsname{\def\PY@tc##1{\textcolor[rgb]{0.40,0.40,0.40}{##1}}}
\expandafter\def\csname PY@tok@mi\endcsname{\def\PY@tc##1{\textcolor[rgb]{0.40,0.40,0.40}{##1}}}
\expandafter\def\csname PY@tok@il\endcsname{\def\PY@tc##1{\textcolor[rgb]{0.40,0.40,0.40}{##1}}}
\expandafter\def\csname PY@tok@mo\endcsname{\def\PY@tc##1{\textcolor[rgb]{0.40,0.40,0.40}{##1}}}
\expandafter\def\csname PY@tok@ch\endcsname{\let\PY@it=\textit\def\PY@tc##1{\textcolor[rgb]{0.25,0.50,0.50}{##1}}}
\expandafter\def\csname PY@tok@cm\endcsname{\let\PY@it=\textit\def\PY@tc##1{\textcolor[rgb]{0.25,0.50,0.50}{##1}}}
\expandafter\def\csname PY@tok@cpf\endcsname{\let\PY@it=\textit\def\PY@tc##1{\textcolor[rgb]{0.25,0.50,0.50}{##1}}}
\expandafter\def\csname PY@tok@c1\endcsname{\let\PY@it=\textit\def\PY@tc##1{\textcolor[rgb]{0.25,0.50,0.50}{##1}}}
\expandafter\def\csname PY@tok@cs\endcsname{\let\PY@it=\textit\def\PY@tc##1{\textcolor[rgb]{0.25,0.50,0.50}{##1}}}

\def\PYZbs{\char`\\}
\def\PYZus{\char`\_}
\def\PYZob{\char`\{}
\def\PYZcb{\char`\}}
\def\PYZca{\char`\^}
\def\PYZam{\char`\&}
\def\PYZlt{\char`\<}
\def\PYZgt{\char`\>}
\def\PYZsh{\char`\#}
\def\PYZpc{\char`\%}
\def\PYZdl{\char`\$}
\def\PYZhy{\char`\-}
\def\PYZsq{\char`\'}
\def\PYZdq{\char`\"}
\def\PYZti{\char`\~}
% for compatibility with earlier versions
\def\PYZat{@}
\def\PYZlb{[}
\def\PYZrb{]}
\makeatother


    % Exact colors from NB
    \definecolor{incolor}{rgb}{0.0, 0.0, 0.5}
    \definecolor{outcolor}{rgb}{0.545, 0.0, 0.0}



    
    % Prevent overflowing lines due to hard-to-break entities
    \sloppy 
    % Setup hyperref package
    \hypersetup{
      breaklinks=true,  % so long urls are correctly broken across lines
      colorlinks=true,
      urlcolor=urlcolor,
      linkcolor=linkcolor,
      citecolor=citecolor,
      }
    % Slightly bigger margins than the latex defaults
    
    \geometry{verbose,tmargin=1in,bmargin=1in,lmargin=1in,rmargin=1in}
    
    

    \begin{document}
    
    
    \maketitle
    
    

    
    \hypertarget{post-requests}{%
\subsection{Post Requests}\label{post-requests}}

\begin{itemize}
\tightlist
\item
  Write post routes, and test with Postman
\item
  Use a form to send a post request
\item
  Use body parser to get form data.
\end{itemize}

    \hypertarget{writing-post-requests}{%
\subsubsection{Writing Post Requests}\label{writing-post-requests}}

\begin{quote}
The post route (post request) sends data to that is either added to a
database, or otherwise being used on the server-side.
\end{quote}

\begin{quote}
Use a POST route when adding data to database, \{a form is submitted,
new user added, creating a new comment or post, etc\}.
\end{quote}

Example:

\hypertarget{in-the-app.js-file}{%
\paragraph{\texorpdfstring{In the \texttt{app.js}
file:}{In the app.js file:}}\label{in-the-app.js-file}}

\begin{Shaded}
\begin{Highlighting}[]

\CommentTok{// Render an ejs template with a list of friends.}
\VariableTok{app}\NormalTok{.}\AttributeTok{get}\NormalTok{(}\StringTok{"/friends"}\OperatorTok{,} \KeywordTok{function}\NormalTok{(req}\OperatorTok{,}\NormalTok{ res) }\OperatorTok{\{}
  \KeywordTok{var}\NormalTok{ friends  }\OperatorTok{=}\NormalTok{ [}\StringTok{"all"}\OperatorTok{,} \StringTok{"my"}\OperatorTok{,} \StringTok{"friends"}\OperatorTok{,} \StringTok{"are"}\OperatorTok{,} \StringTok{"dead"}\NormalTok{]}\OperatorTok{;}
  \VariableTok{res}\NormalTok{.}\AttributeTok{render}\NormalTok{(}\StringTok{"friends"}\OperatorTok{,} \OperatorTok{\{}\DataTypeTok{friends}\OperatorTok{:}\NormalTok{friends}\OperatorTok{\}}\NormalTok{)}\OperatorTok{;}
\OperatorTok{\}}\NormalTok{)}\OperatorTok{;}

\CommentTok{// The post route will look for a form whose action is "/addfriend",}
\CommentTok{//   and run res.send() when/if it finds a matching form.}
\CommentTok{// Otherwise, an error is thrown/displayed on the webpage.}
\CommentTok{//   Usually, it's "Cannot POST '/url' ".}
\VariableTok{app}\NormalTok{.}\AttributeTok{post}\NormalTok{(}\StringTok{"/addfriend"}\OperatorTok{,} \KeywordTok{function}\NormalTok{(req}\OperatorTok{,}\NormalTok{ res) }\OperatorTok{\{}
  \VariableTok{res}\NormalTok{.}\AttributeTok{send}\NormalTok{(}\StringTok{"WE IN HERE (the post route)!}\SpecialCharTok{\textbackslash{}n}\StringTok{ We live baby, yeah!"}\NormalTok{)}\OperatorTok{;}
\OperatorTok{\}}\NormalTok{)}\OperatorTok{;}
\end{Highlighting}
\end{Shaded}

\hypertarget{in-the-friends.ejs-file}{%
\paragraph{\texorpdfstring{In the \texttt{friends.ejs}
file}{In the friends.ejs file}}\label{in-the-friends.ejs-file}}

\begin{Shaded}
\begin{Highlighting}[]
\KeywordTok{<h1>}\NormalTok{FRIENDS GO HERE}\KeywordTok{</h1>}

\CommentTok{<!-- List all friends-->}
\ErrorTok{<}\NormalTok{% friends.forEach(function(friend)\{ %>}
  \KeywordTok{<li>}\ErrorTok{<}\NormalTok{%= friend %>}\KeywordTok{</li>}
\ErrorTok{<}\NormalTok{% \}); %>}

\CommentTok{<!-- action is the url, method is the type of request \{GET, POST,etc\}-->}
\CommentTok{<!-- "action" in the ejs file MUST match the url (the first arg of app.post()) of the post route in the app.js file-->}
\KeywordTok{<form}\OtherTok{ action=}\StringTok{"/addfriend"}\OtherTok{ method=}\StringTok{"POST"}\KeywordTok{>}
    
  \CommentTok{<!--When this input is submitted, the post route will look for the form -->}
  \CommentTok{<!--whose action is "/addfriend", and run (the code in) the post route.-->}
  \KeywordTok{<input}\OtherTok{ type=}\StringTok{"text"}\OtherTok{ placeholder=}\StringTok{"name"}\KeywordTok{></input>}
    
  \CommentTok{<!-- (apparently submit is the default button type)-->}
  \KeywordTok{<button>}\NormalTok{MAKE NEW FRIEND}\KeywordTok{</button>}
\KeywordTok{</form>}
\end{Highlighting}
\end{Shaded}

    \hypertarget{getting-form-data-out-of-form}{%
\subsubsection{Getting form data out of
form}\label{getting-form-data-out-of-form}}

\hypertarget{in-the-app.js-file.}{%
\paragraph{\texorpdfstring{In the \texttt{app.js}
file.}{In the app.js file.}}\label{in-the-app.js-file.}}

\begin{Shaded}
\begin{Highlighting}[]
\KeywordTok{var}\NormalTok{ express }\OperatorTok{=} \AttributeTok{require}\NormalTok{(}\StringTok{"express"}\NormalTok{)}\OperatorTok{;}
\KeywordTok{var}\NormalTok{ app }\OperatorTok{=} \AttributeTok{express}\NormalTok{()}


\CommentTok{// remember to npm install body-parser too!}
\CommentTok{// bodyParser parses the body of the (POST) request into a JS object.}
\CommentTok{// Express must EXPLICITLY know that body-parser is being used.}
\KeywordTok{var}\NormalTok{ bodyParser }\OperatorTok{=} \AttributeTok{require}\NormalTok{(}\StringTok{"body-parser"}\NormalTok{)}\OperatorTok{;}

\CommentTok{// @Steeve: test if this is still needed to work.}
\CommentTok{// @Steeve: maybe check the body-parser docs for }
\CommentTok{//   what urlencoded(\{extended:true\} does.}
\CommentTok{// Basically a copypasta to retrieve the data from a form.}
\VariableTok{app}\NormalTok{.}\AttributeTok{use}\NormalTok{(}\VariableTok{bodyParser}\NormalTok{.}\AttributeTok{urlencoded}\NormalTok{(}\OperatorTok{\{}\DataTypeTok{extended}\OperatorTok{:}\KeywordTok{true}\OperatorTok{\}}\NormalTok{))}\OperatorTok{;}

\KeywordTok{var}\NormalTok{ friends  }\OperatorTok{=}\NormalTok{ [}\StringTok{"all"}\OperatorTok{,} \StringTok{"my"}\OperatorTok{,} \StringTok{"friends"}\OperatorTok{,} \StringTok{"are"}\OperatorTok{,} \StringTok{"dead"}\NormalTok{]}\OperatorTok{;}


\CommentTok{// app.use(express.static("public"));}
\VariableTok{app}\NormalTok{.}\AttributeTok{set}\NormalTok{(}\StringTok{"view engine"}\OperatorTok{,} \StringTok{"ejs"}\NormalTok{)}\OperatorTok{;}

\VariableTok{app}\NormalTok{.}\AttributeTok{get}\NormalTok{(}\StringTok{"/"}\OperatorTok{,} \KeywordTok{function}\NormalTok{(req}\OperatorTok{,}\NormalTok{ res) }\OperatorTok{\{}
  \VariableTok{res}\NormalTok{.}\AttributeTok{render}\NormalTok{(}\StringTok{"home"}\NormalTok{)}\OperatorTok{;}
  
\OperatorTok{\}}\NormalTok{)}

 
\CommentTok{// Render an ejs template with a list of friends.}
\VariableTok{app}\NormalTok{.}\AttributeTok{get}\NormalTok{(}\StringTok{"/friends"}\OperatorTok{,} \KeywordTok{function}\NormalTok{(req}\OperatorTok{,}\NormalTok{ res) }\OperatorTok{\{}
  
  \CommentTok{// REMEMBER: friends  = ["all", "my", "friends", "are", "dead"];}
  \VariableTok{res}\NormalTok{.}\AttributeTok{render}\NormalTok{(}\StringTok{"friends"}\OperatorTok{,} \OperatorTok{\{}\DataTypeTok{friends}\OperatorTok{:}\NormalTok{friends}\OperatorTok{\}}\NormalTok{)}\OperatorTok{;}
\OperatorTok{\}}\NormalTok{)}\OperatorTok{;}


\CommentTok{// The post route will look for a form whose action is "/addfriend",}
\CommentTok{//   and run res.send() when/if it finds a matching form.}
\CommentTok{// Otherwise, an error is thrown/displayed on the webpage.}
\CommentTok{//   Usually, it's "Cannot POST '/url' ".}
\VariableTok{app}\NormalTok{.}\AttributeTok{post}\NormalTok{(}\StringTok{"/addfriend"}\OperatorTok{,} \KeywordTok{function}\NormalTok{(req}\OperatorTok{,}\NormalTok{ res) }\OperatorTok{\{}
  \KeywordTok{var}\NormalTok{ newFriend }\OperatorTok{=} \VariableTok{req}\NormalTok{.}\VariableTok{body}\NormalTok{.}\AttributeTok{name}\OperatorTok{;}
  
  \VariableTok{friends}\NormalTok{.}\AttributeTok{push}\NormalTok{(newFriend)}\OperatorTok{;}
  
  \CommentTok{// takes the name of a route (the url of the particular page),}
  \CommentTok{//  and runs the code in that route.}
  \VariableTok{res}\NormalTok{.}\AttributeTok{redirect}\NormalTok{(}\StringTok{"/friends"}\NormalTok{)}\OperatorTok{;}
\OperatorTok{\}}\NormalTok{)}\OperatorTok{;}

\VariableTok{app}\NormalTok{.}\AttributeTok{listen}\NormalTok{(}\VariableTok{process}\NormalTok{.}\VariableTok{env}\NormalTok{.}\AttributeTok{PORT}\OperatorTok{,} \VariableTok{process}\NormalTok{.}\VariableTok{env}\NormalTok{.}\AttributeTok{IP}\OperatorTok{,} \KeywordTok{function}\NormalTok{() }\OperatorTok{\{}
  \VariableTok{console}\NormalTok{.}\AttributeTok{log}\NormalTok{(}\StringTok{"Server has started!"}\NormalTok{)}\OperatorTok{;}
\OperatorTok{\}}\NormalTok{)}
\end{Highlighting}
\end{Shaded}

\hypertarget{in-the-friends.ejs-file.}{%
\paragraph{\texorpdfstring{In the \texttt{friends.ejs}
file.}{In the friends.ejs file.}}\label{in-the-friends.ejs-file.}}

\begin{Shaded}
\begin{Highlighting}[]
\KeywordTok{<h1>}\NormalTok{FRIENDS GO HERE}\KeywordTok{</h1>}

\CommentTok{<!-- List all friends-->}
\ErrorTok{<}\NormalTok{% friends.forEach(function(friend)\{ %>}
  \KeywordTok{<li>}\ErrorTok{<}\NormalTok{%= friend %>}\KeywordTok{</li>}
\ErrorTok{<}\NormalTok{% \}); %>}

\CommentTok{<!-- action is the url, method is the type of request \{GET, POST,etc\}-->}
\CommentTok{<!-- "action" in the ejs file MUST match --><!--the url (the first arg of app.post()) of the post route in the app.js file-->}
\KeywordTok{<form}\OtherTok{ action=}\StringTok{"/addfriend"}\OtherTok{ method=}\StringTok{"POST"}\KeywordTok{>}
  
  \CommentTok{<!-- "name" is the value of the form, and will be looked for in the route  -->}
  \CommentTok{<!-- name is one of the properties that will be sent back in the body of the request.-->}
  \KeywordTok{<input}\OtherTok{ type=}\StringTok{"text"}\OtherTok{ name=}\StringTok{"name"}\OtherTok{ placeholder=}\StringTok{"name"}\KeywordTok{></input>}
  
  \CommentTok{<!--When the above input is submitted, the post route will look for the form -->}
  \CommentTok{<!--whose action is "/addfriend", and run (the code in) the post route.-->}

    
  \CommentTok{<!-- (apparently submit is the default button type)-->}
  \KeywordTok{<button>}\NormalTok{MAKE NEW FRIEND}\KeywordTok{</button>}
\KeywordTok{</form>}
\end{Highlighting}
\end{Shaded}

    \hypertarget{express-boilerplate-v4}{%
\subsection{Express Boilerplate, v4}\label{express-boilerplate-v4}}

Steps 1. Create Project Directory and \texttt{cd} into it. 2. Run
\texttt{npm\ init} wizard for a bunch of setup stuff (including
dependencies setup). * Important: this line must go before installing
something with the \texttt{-\/-save} flag. * follow the directions pls *
\texttt{init} should generate a package.json file.

\begin{enumerate}
\def\labelenumi{\arabic{enumi}.}
\setcounter{enumi}{2}
\item
  Run \texttt{npm\ install\ express\ -\/-save} to install express, and
  to add express to the dependencies.

  \begin{itemize}
  \tightlist
  \item
    \textbf{NOTE:} A package-losck.json might appear, it is safe to
    commit. It \textbf{SHOULD} be committed.
  \item
    \textbf{NOTE:} a node\_modules folder will probably be added, this
    is OK.
  \end{itemize}
\item
  Run \texttt{npm\ install\ ejs\ -\/-save} to install ejs (for the
  embedded JavaScript files).
\item
  Run \texttt{npm\ install\ body-parser\ -\/-save} to install
  body-parser (to get extract form data for our POST requests).
\item
  \texttt{touch} the app.js file (while inside your Project Directory).
\item
  Inside the app.js file, make sure it looks like this to start:

\begin{Shaded}
\begin{Highlighting}[]
 \KeywordTok{var}\NormalTok{ express }\OperatorTok{=} \AttributeTok{require}\NormalTok{(}\StringTok{"express"}\NormalTok{)}\OperatorTok{;}
 \KeywordTok{var}\NormalTok{ app }\OperatorTok{=} \AttributeTok{express}\NormalTok{()}\OperatorTok{;}

 \CommentTok{// remember to npm install body-parser too!}
 \CommentTok{// bodyParser parses the body of the (POST) request into a JS object.}
 \CommentTok{// Express must EXPLICITLY know that body-parser is being used.}
 \KeywordTok{var}\NormalTok{ bodyParser }\OperatorTok{=} \AttributeTok{require}\NormalTok{(}\StringTok{"body-parser"}\NormalTok{)}\OperatorTok{;} 

 \CommentTok{// @Steeve: test if this is still needed to work.}
 \CommentTok{// @Steeve: maybe check the body-parser docs for}
 \CommentTok{//   what urlencoded(\{extended:true\} does.}
 \CommentTok{// Basically a copypasta to retrieve the data from a form.}
 \VariableTok{app}\NormalTok{.}\AttributeTok{use}\NormalTok{(}\VariableTok{bodyParser}\NormalTok{.}\AttributeTok{urlencoded}\NormalTok{(}\OperatorTok{\{}\DataTypeTok{extended}\OperatorTok{:}\KeywordTok{true}\OperatorTok{\}}\NormalTok{))}\OperatorTok{;}

 \CommentTok{// tell express to serve the "ProjectName/public/" folder.}
 \VariableTok{app}\NormalTok{.}\AttributeTok{use}\NormalTok{(}\VariableTok{express}\NormalTok{.}\AttributeTok{static}\NormalTok{(}\StringTok{"public"}\NormalTok{))}\OperatorTok{;}

 \CommentTok{// explicitly tell express to be able to use ejs files}
 \CommentTok{// allows us to write res.render("filename.ejs");}
 \CommentTok{//  as just res.render("filename");}
 \VariableTok{app}\NormalTok{.}\AttributeTok{set}\NormalTok{(}\StringTok{"view engine"}\OperatorTok{,} \StringTok{"ejs"}\NormalTok{)}\OperatorTok{;}

 \CommentTok{// render the home.ejs file. Don't need .ejs extension IF app.set("view engine", "ejs") is used.}
 \VariableTok{app}\NormalTok{.}\AttributeTok{get}\NormalTok{(}\StringTok{"/"}\OperatorTok{,} \KeywordTok{function}\NormalTok{(req}\OperatorTok{,}\NormalTok{ res)}\OperatorTok{\{}
 \VariableTok{res}\NormalTok{.}\AttributeTok{render}\NormalTok{(}\StringTok{"home"}\NormalTok{)}\OperatorTok{;}
 \OperatorTok{\}}\NormalTok{)}\OperatorTok{;}

 \CommentTok{// The post route will look for a form whose action is "/doSomething",}
 \CommentTok{//   and run res.send() when/if it finds a matching form.}
 \CommentTok{// Otherwise, an error is thrown/displayed on the webpage.}
 \CommentTok{//   Usually, it's "Cannot POST '/url' ".}
 \VariableTok{app}\NormalTok{.}\AttributeTok{post}\NormalTok{(}\StringTok{"/doSomething"}\OperatorTok{,} \KeywordTok{function}\NormalTok{(req}\OperatorTok{,}\NormalTok{ res) }\OperatorTok{\{}
   \CommentTok{// Example of accessing a field from the request body.}
   \CommentTok{// requires body-parser}
   \KeywordTok{var}\NormalTok{ task }\OperatorTok{=} \VariableTok{req}\NormalTok{.}\VariableTok{body}\NormalTok{.}\AttributeTok{task}\OperatorTok{;}

   \CommentTok{// takes the name of a route (the url of the particular page),}
   \CommentTok{//  and runs the code in that route.}
   \CommentTok{// So redirect the user to homepage.}
   \VariableTok{res}\NormalTok{.}\AttributeTok{redirect}\NormalTok{(}\StringTok{"/friends"}\NormalTok{)}\OperatorTok{;}
 \OperatorTok{\}}\NormalTok{)}\OperatorTok{;}

 \CommentTok{//OPTIONAL: You may want to add a splat route matcher for an error page}
 \CommentTok{// MUST be the last route in the file though.}
 \VariableTok{app}\NormalTok{.}\AttributeTok{get}\NormalTok{(}\StringTok{"*"}\OperatorTok{,} \KeywordTok{function}\NormalTok{(req}\OperatorTok{,}\NormalTok{ res)}\OperatorTok{\{}
 \VariableTok{res}\NormalTok{.}\AttributeTok{send}\NormalTok{(}\StringTok{"ERROR 404: PAGE NOT FOUND!"}\NormalTok{)}\OperatorTok{;}
 \OperatorTok{\}}\NormalTok{)}\OperatorTok{;}

 \CommentTok{// Make sure to set your app to listen, right at the very end of the app.js file.}
 \CommentTok{// tHiS iS wHat RuNs ThE sErVeR.}
 \CommentTok{// Also, }\AlertTok{NOTE}\CommentTok{: things starting w/ "process.env.*" are only used on Cloud9.}
 \CommentTok{//   When using an actual server,make sure to put its real port and IP address.}
 \VariableTok{app}\NormalTok{.}\AttributeTok{listen}\NormalTok{(}\VariableTok{process}\NormalTok{.}\VariableTok{env}\NormalTok{.}\AttributeTok{PORT}\OperatorTok{,} \VariableTok{process}\NormalTok{.}\VariableTok{env}\NormalTok{.}\AttributeTok{IP}\OperatorTok{,} \KeywordTok{function}\NormalTok{()}\OperatorTok{\{}
 \VariableTok{console}\NormalTok{.}\AttributeTok{log}\NormalTok{(}\StringTok{"Now serving your app!"}\NormalTok{)}\OperatorTok{;}
 \OperatorTok{\}}\NormalTok{)}\OperatorTok{;}
\end{Highlighting}
\end{Shaded}

  \textbf{NOTE:} the \texttt{.} (dots) represent where the majority of
  user written code (routes, etc) will go.
\item
  Create the ``ProjectDirectory/public/'' folder (for assets like
  stylesheets and scripts) and the ``ProjectDirectory/views/'' folder to
  hold the ejs templates that will be served.
\item
  \texttt{touch} the public/app.css or (``public/stylesheets/app.css''
  if you're responsible) file and write your vanilla CSS in there.
\item
  Create the ``ProjectDirectory/views/partials'' directory (sometimes
  layouts is used instead of partials) to the boilerplate elements.
\item
  Create your ``ProjectDirectory/views/partials/header.ejs'' and
  ``ProjectDirectory/views/partials/footer.ejs'' files for the top and
  bottom halves of the HTML bolierplate, respectively.
\item
  Inside all your served EJS files (served ejs files \textbf{must} be
  inside ``ProjectDirectory/views/'' folder), make sure it looks like
  this to start:
\end{enumerate}

\begin{Shaded}
\begin{Highlighting}[]

 \CommentTok{// full filepath: "ProjecDirectory/views/partials/header.ejs"}
 \OperatorTok{<%}\NormalTok{ include partials/header }\OperatorTok\NormalTok{ include partials/footer }\OperatorTok{%>}
 
\end{Highlighting}
\end{Shaded}

Usually, views/partials/header.ejs template looks something like:

```html \textless{}!DOCTYPE html\textgreater{}

``` * (The top half of the HTML boilerplate.)

Usually, view/partials/footer.ejs template looks something like:
```javascript

\begin{verbatim}
<p>Copyright 2018</p>
  </body>
</html>

```
 * (The bottom half of the HTML boilerplate.)
\end{verbatim}


    % Add a bibliography block to the postdoc
    
    
    
    \end{document}
